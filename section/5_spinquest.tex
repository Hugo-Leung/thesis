\documentclass[../main.tex]{subfiles}
\begin{document}

\ifSubfilesClassLoaded{
	\mainmatter
	\setcounter{chapter}{4}
}{}

\chapter{TMD with Transversely Polarized Target}
\label{ch:spinquest}

\section{Introduction}
SpinQuest is a follow-up experiment of SeaQuest \cite{brown2014}. The same beam 
line and spectrometer used in SeaQuest, described in \cref{M-ch:seaquest},
will be used. The target would be replaced by transversely polarized \ce{NH_3}
and \ce{ND_3} targets. The goal is to measure the azimuthal asymmetry in 
Drell-Yan process and $J/\Psi$ production, which can provide information on the
the transverse motion of the partons in the nucleons.

\subsection{Transverse momentum dependent parton distributions}

\section{SpinQuest Experiment}

\subsection{Polarized Target}
The polarized target was previously used by EXXX. \pdfmargincomment{which experiment?
	Is it  E143 at SLAC?} It has been rebuilt and tested by the University of
Virginia group. The target consists of a \SI{5}{\tesla} superconducting
split coil magnet, a \ce{^4He}  evaporation refrigerator, a \SI{140}{\GHz}
microwave source and a \SI{15000}{\cubic\meter\per\hour} pumping system. The
target is polarized using Dynamic Nuclear Polarization(DNP) \cite{crabb1995}.
\pdfmargincomment{should there be a brief explanation on DNP?}

\subsection{Data Acquisition System}

\section{Preliminary Result}
\pdfmargincomment{Hoping we got a mass spectrum from spinquest and possibly comment
	on the enhance jpsi peak due to new trigger road and magnet settings}

\ifSubfilesClassLoaded{ \printbibliography[heading=bibintoc,title={References}]}{}
\end{document}
