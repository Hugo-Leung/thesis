\documentclass[../main.tex]{subfiles}
\begin{document}

\ifSubfilesClassLoaded{
	\mainmatter
	\setcounter{chapter}{2}
}{}

\chapter{Analysis}
\label{ch:analysis}

\section{Data Sets}

\begin{table}[h!]
	\centering
	\begin{tabular}{ c c c }
		\hline
		Run & Experimental Conditions & Dates                    \\
		\hline
		2   & Roadset 57              & 06/25/2014 to 08/20/2014 \\
		    & Roadset 59              & 08/20/2014 to 09/03/2014 \\
		\hline
		N/A & D3p and D3m moved       & 10//03/2014              \\
		\hline
		3   & Roadset 62              & 11/08/2014 to 01/14/2015 \\
		    & Deuterium Change        & 11/13/2014               \\
		    & Deuterium Change        & 12/02/2014               \\
		    & Magnet Polarity flipped & 01/14/2015               \\
		    & Roadset 67              & 01/25/2015 to 06/19/2015 \\
		    & Deuterium Change        & 04/24/2014               \\
		    & D1 and H1 moved         & 05/13/2015               \\
		    & Roadset 70              & 06/19/2015 to 07/03/2015 \\
		\hline
		4   & Constant adjustments    & 11/13/2015 to 03/06/2016 \\
		5   & Roadset 78              & 03/06/2016 to 07/29/2016 \\
		6   &                         & 01/14/2017 to 07/07/2017 \\
	\end{tabular}
	\caption{SeaQuest data sets and apparatus adjustments}
	\label{tab:dataset}
\end{table}

\section{Monte Carlo}

\section{Track Reconstruction}

\section{Event Selection}

\section{Target Contamination}

\section{Intensity Extrapolation}
\label{sec:extrapolation}

\section{Mass Spectrum Fitting}
\pdfmargincomment{see analysis note}
\subsection{\texorpdfstring{$p_T$}{pT} re-weighting in Monte Carlo}

\section{Acceptance Calculation}

\section{Tracker Efficiency}

\section{Drell-Yan NLO calculation}
The next-to-leading order (NLO) calculation is done using a parton level Monte 
Carlo program written by the INFN group\cite{catani2009,catani2007}.

\ifSubfilesClassLoaded{ \printbibliography[heading=bibintoc,title={References}]}{}
\end{document}
