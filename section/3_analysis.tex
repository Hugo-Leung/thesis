\documentclass[../main.tex]{subfiles}
\begin{document}

\ifSubfilesClassLoaded{
	\mainmatter
	\setcounter{chapter}{2}
}{}

\chapter{Analysis}
\label{ch:analysis}

\section{Data Sets}

\begin{table}[h!]
	\centering
	\begin{tabular}{ c c c }
		\hline
		Run & Experimental Conditions & Dates                    \\
		\hline
		2   & Roadset 57              & 06/25/2014 to 08/20/2014 \\
		    & Roadset 59              & 08/20/2014 to 09/03/2014 \\
		\hline
		N/A & D3p and D3m moved       & 10//03/2014              \\
		\hline
		3   & Roadset 62              & 11/08/2014 to 01/14/2015 \\
		    & Deuterium Change        & 11/13/2014               \\
		    & Deuterium Change        & 12/02/2014               \\
		    & Magnet Polarity flipped & 01/14/2015               \\
		    & Roadset 67              & 01/25/2015 to 06/19/2015 \\
		    & Deuterium Change        & 04/24/2014               \\
		    & D1 and H1 moved         & 05/13/2015               \\
		    & Roadset 70              & 06/19/2015 to 07/03/2015 \\
		\hline
		4   & Constant adjustments    & 11/13/2015 to 03/06/2016 \\
		5   & Roadset 78              & 03/06/2016 to 07/29/2016 \\
		6   &                         & 01/14/2017 to 07/07/2017 \\
	\end{tabular}
	\caption{SeaQuest data sets and apparatus adjustments}
	\label{tab:dataset}
\end{table}

\section{Monte Carlo}

\section{Track Reconstruction}

\section{Event Selection}
\begin{table}[ht!]
	\centering
	\caption{Track level cuts.}
	\label{table:trackCut}
	\begin{tabular}{|m{4.5cm}|m{7cm}|m{3cm}|}
		\hline
		Variable                                                                                                                             & Description                                                                          & Cut                          \\ \hline
		$\chi^2_{target}$                                                                                                                    & $\chi^2$ when tack is   forced to pass through center of target ($z=\SI{-129}{\cm}$) & $< 15$                       \\ \hline
		$pz_1$                                                                                                                               & z momentum at station 1                                                              & (\SI{9}{\GeV},\SI{75}{\GeV}) \\ \hline
		nHits                                                                                                                                & total number of hits on wire chambers by each   muon track                           & $> 13$                       \\ \hline
		$x_T^2 +(y_T - \textrm{beamOffset})^2$                                                                                               & radial distance of track from   beam line at the center of target                    & $< \SI{320}{\cm\squared}$    \\ \hline
		$x_D^2 +(y_D - \textrm{beamOffset})^2$                                                                                               &
		radial distance of track from   beam line at the center of beam dump ($z=\SI{42}{\cm}$)                                              &
		(\SI{8}{\cm\squared},\SI{1100}{\cm\squared})   \footnotemark[1]                                                                                                                                                                                            \\ \hline
		\begin{tabular}[c]{@{}c@{}}$\chi^2_{target}<1.5\chi^2_{upstream}$\\      $\chi^2_{target}<1.5\chi^2_{dump}$\end{tabular}             &
		$\chi^2$ when tack is   forced to pass through $z=\SI{-490}{\cm}$(upstream), $z=\SI{-129}{\cm}$(traget) and   $z=\SI{42}{\cm}$(dump) &
		\\ \hline
		$z_0$                                                                                                                                & z position of track's closest approach to beam   line                                & (\SI{320}{\cm},\SI{-5}{\cm}) \\ \hline
		$\chi^2/(\textrm{nHits}-5)$                                                                                                          & $\chi^2/\textrm{ndf}$                                                                & $<12$                        \\ \hline
		$y_1/y_3$                                                                                                                            & y position of track ast St.\ 1   and St.\ 3                                          & $<1$                         \\ \hline
		$y_1\times y_3$                                                                                                                      &                                                                                      & $>0$                         \\ \hline
		$| |px_1 - px_3| -0.416|$                                                                                                            & difference in x momentum at St.\   1 and St.\ 3 accounting for the Kmag kick         & $<\SI{0.008}{\GeV}$          \\ \hline
		$|py_1 - py_3|$                                                                                                                      & difference in y momentum at St.\   1 and St.\ 3                                      & $<\SI{0.008}{\GeV}$          \\ \hline
		$|pz_1 - pz_3|$                                                                                                                      & difference in z momentum at St.\   1 and St.\ 3                                      & $<\SI{0.08}{\GeV}$           \\ \hline
		$|py_1 |$                                                                                                                            & absolute value of the y momentum   at St.\ 1                                         & $>\SI{0.02}{\GeV}$           \\ \hline
	\end{tabular}
\end{table}
\begin{table}[ht!]
	\centering
	\caption{dimuon level cuts.}
	\label{table:dimuonCut}
	\begin{tabular}{|m{4.5cm}|m{7cm}|m{3cm}|}
		\hline
		Variable                                                                                            & Description                                                                              & Cut                            \\ \hline
		$|dx|$                                                                                              & x position of dimuon vertex                                                              & $<\SI{0.25}{\cm}$              \\ \hline
		$|dy-\textrm{beamOffset}|$                                                                          & y position of dimuon vertex                                                              & $<\SI{0.22}{\cm}$              \\ \hline
		$dz$                                                                                                & z position of dimuon vertex                                                              & (\SI{-280}{\cm},\SI{-5}{\cm})  \\ \hline
		$dx^2+(dy-\textrm{beamOffset})^2$                                                                   & radial distance of the dimuon vertex from beam line                                      & $<\SI{0.06}{\cm\squared}$      \\ \hline
		$|dpx|$                                                                                             & absolute value of dimuon x   momentum                                                    & $<\SI{1.8}{\GeV}$              \\ \hline
		$|dpy|$                                                                                             & absolute value of dimuon y   momentum                                                    & $<\SI{2}{\GeV}$                \\ \hline
		$dpx^2+dpy^2$                                                                                       & transverse momentum squared of   dimuon                                                  & $<\SI{5}{\GeV}$                \\ \hline
		dpz                                                                                                 & dimuon z momentum                                                                        & (\SI{38}{\GeV},\SI{116}{\GeV}) \\ \hline
		$|\textrm{tackSeparation}|$                                                                         & distance in z between points   of  closest approach between $\mu^+$   and $\mu^-$ track  & $<\SI{270}{\cm}$               \\ \hline
		$\chi^2_{dimuon}$                                                                                   & $\chi^2$ when both $\mu^+$ and   $\mu^-$ tracks are forced to pass thorugh dimuon vertex & $<18$                          \\ \hline
		$\begin{aligned} |\chi^2_{target}(\mu^+) &+ \chi^2_{target}(\mu^-)\\& -x^2_{dimuon}| \end{aligned}$ &                                                                                          & $<2$                           \\ \hline
		$y_3(\mu^+) \times y_3(\mu^-)$                                                                      & y position at St.\ 3 for both tracks                                                     & $<0$                           \\ \hline
		$\textrm{nHists}(\mu^+)+\textrm{nHists}(\mu^-)$                                                     & sum of the total number if hits   on wire chamber by $\mu^+$ and $\mu^-$ track           & $>29$                          \\ \hline
		$\textrm{nHists}_1(\mu^+)+\textrm{nHists}_1(\mu^-)$                                                 & sum of the total number if hits   on St.\ 1 wire chamber by $\mu^+$ and $\mu^-$ track    & $>8$                           \\ \hline
		$|x_1(\mu^+) + x_1(\mu^-)|$                                                                         & sum off x position of of tracks   at St.\ 1                                              & $>\SI{42}{\cm}$                \\ \hline
	\end{tabular}
\end{table}
\begin{table}[h!]
	\centering
	\caption{ physics cuts}
	\label{table:physCut}
	\begin{tabular}{|m{4.5cm}|m{7cm}|m{3cm}|}
		\hline
		Variable         & Description                          & Cut                                                            \\ \hline
		mass             & dimuon mass                          & (\SI{2}{\GeV},\SI{8.8}{\GeV}) \footnotemark[1]\footnotemark[2] \\ \hline
		$x_F$            & Feynman x                            & (-0.1,0.95)                                                    \\ \hline
		$x_{target}$     & Bjorken x of target                  & (0.005,0.55)  \footnotemark[1]                                 \\ \hline
		$|\cos(\theta)|$ & polar angle in Collins-Soper   frame & $<0.5$                                                         \\ \hline
	\end{tabular}
\end{table}
\begin{table}[h!]
	\centering
	\caption{occupancy cuts}
	\label{table:occCut}
	\begin{tabular}{|m{4.5cm}|m{7cm}|m{3cm}|}
		\hline
		Variable                          & Description                                & Cut      \\ \hline
		D1                                & total hits on St.\ 1 Chamber               & $<400$   \\ \hline
		D2                                & total hits on St.\ 2 Chamber               & $<400$   \\ \hline
		D3                                & total hits on St.\ 3 Chamber               & $<400$   \\ \hline
		$D1+D2+D3$                        & total numbers of hits in all wire Chambers & $<1000$  \\ \hline
		Trigger intensity\footnotemark[3] & numbers of proton in the triggering bucket & $<80000$ \\ \hline
	\end{tabular}
\end{table}

\footnotetext[1]{modified from Ref.\ \cite{chuck-2111}}
\footnotetext[2]{The 0.99 scaling factor is applied to MC}
\footnotetext[3]{Not applied to MC or Mix background}


\section{Target Contamination}

\section{Intensity Extrapolation}
\label{sec:extrapolation}

\section{Mass Spectrum Fitting}
\pdfmargincomment{see analysis note}
\subsection{\texorpdfstring{$p_T$}{pT} re-weighting in Monte Carlo}

\section{Acceptance Calculation}

\section{Tracker Efficiency}

\section{Drell-Yan NLO calculation}
The next-to-leading order (NLO) calculation is done using a parton level Monte 
Carlo program written by the INFN group\cite{catani2009,catani2007}.

\ifSubfilesClassLoaded{ \printbibliography[heading=bibintoc,title={References}]}{}
\end{document}
