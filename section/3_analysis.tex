\documentclass[../main.tex]{subfiles}
\begin{document}

\ifSubfilesClassLoaded{
	\mainmatter
	\setcounter{chapter}{4}
}{}

\chapter{Analysis}
\label{ch:analysis}

\section{Data Sets}
SeaQuest received its first proton in the beginning of 2012, and began beam
commission. After a short commissioning run, the Main Injector was shut down for
upgrade. After the Main Injector restarted in 2013, the data collection began in
2014. The timeline and some important dates is tabulated in \cref{tab:dataset}.
\begin{table}[h!]
	\centering
	\begin{tabular}{ c c c }
		\hline
		Run & Experimental Conditions & Dates                    \\
		\hline
		2   & Roadset 57              & 06/25/2014 to 08/20/2014 \\
		    & Roadset 59              & 08/20/2014 to 09/03/2014 \\
		\hline
		N/A & D3p and D3m moved       & 10//03/2014              \\
		\hline
		3   & Roadset 62              & 11/08/2014 to 01/14/2015 \\
		    & Deuterium Change        & 11/13/2014               \\
		    & Deuterium Change        & 12/02/2014               \\
		    & Magnet Polarity flipped & 01/14/2015               \\
		    & Roadset 67              & 01/25/2015 to 06/19/2015 \\
		    & Deuterium Change        & 04/24/2014               \\
		    & D1 and H1 moved         & 05/13/2015               \\
		    & Roadset 70              & 06/19/2015 to 07/03/2015 \\
		\hline
		4   & Constant adjustments    & 11/13/2015 to 03/06/2016 \\
		5   & Roadset 78              & 03/06/2016 to 07/29/2016 \\
		6   &                         & 01/14/2017 to 07/07/2017 \\
	\end{tabular}
	\caption{SeaQuest data sets and apparatus adjustments}
	\label{tab:dataset}
\end{table}

\section{Monte Carlo}
\label{sec:MC}

One of the changes made to the Monte Carlo generator during this study 
is the maximum transverse momentum of the virtual photon.

The maximum $P_T$ carried by the virtual photon is given by
\begin{equation}
	\left(P_T^{\mathtt{max}}\right)^2 = \frac{s}{4} \left[1-M^2_\gamma/s\right]^2 - P_L^2,
\end{equation}
where $P_L$ is the longitudinal momentum of the virtual photon.
Subsituting $x_F = \frac{P_L}{\sqrt{s}\left[1-M^2_\gamma\right]}$,
\begin{equation}
	\left(P_T^{\textrm{max}}\right)^2 = \frac{s}{4} \left[1-M^2_\gamma/s\right]^2\left[1-x_F^2\right].
	\label{eq:pTmax}
\end{equation}
Earlier versions of the SeaQuest Monte Carlo generators used an inconsistent definition of $x_F$
and $P_T^{\textrm{max}}$ resulting in an incomplete coverage of the kinematic phase space.

\subsection{\texorpdfstring{$P_T$}{P_T} re-weighting in Monte Carlo}
This was first studied by S.~Prasard \cite{prasad2020}.
Because the transverse momentum distribution ($p_T$) cannot be calculated using
fixed order pQCD, it is typically parameterized using some functional form.
In the SeaQuest monte carlo simulation, the input $p_T$ distribution is based on the
Kaplan functional form \cite{kaplan1978}.
\begin{equation}
	f\left(P_T^2\right) \propto \frac{1}{\left(1+ P_T^2/p_1\right)^6},
	\label{eq:kaplan}
\end{equation}
with $p_1$ is set to \SI{2.8}{\GeV}. The parameter $p_1$ controls the broadness
of the $P_T$ distribution. In the MC, the $P_T$ of a dimuon is chosen randomly
according to the \cref{eq:kaplan}. If the chosen $P_T$ is exceed \cref{eq:pTmax}
and does not satisfy the kinematic constrain, a new $P_T$ is will be chosen until
the constrain is satisfy.

The chosen value of $p_1$ is taken from experiments conducted at higher
energy, and may not be suitable to the SeaQuest kinematic. An additional weight is
applied to account for the difference between the $P_T$ distribution in the MC input
and in the real data. In addition, the $P_T$ distribution can also depends on $x_F$.
The $x_F$ dependence of the $P_T$ distribution has been reported in the pion induced
Drell-Yan experiment E615 \cite{conway1989}. The $x_F$ and $\sqrt{s}$ dependence in
proton induced Drell-Yan have also been reported in Ref.~\cite{prasad2020}.

In this analysis, a different re-weighting formula is used. Because of the way $p_T$ is chosen
in the MC, the probability density function is normalized to unity as follows
\begin{equation}
	\int^{\left(P_T^{\textrm{max}}\right)^2}_0 dP_T^2 \frac{N}{\left(1+ P_T^2/p_1\right)^6}=1
\end{equation}
where $N$ is the normalization constant and is given by
\begin{equation}
	N=\frac{5}{p_1^2-p_1^2\left[ 1+ \left(P_T^{\textrm{max}}\right)^2/p_1^2\right]^{-5}}.
\end{equation}
Therefore the $p_T$ reweight is given by
\begin{equation}
	P_T \textrm{ reweight}\left(m,x_F\right)=
	\frac{\left(1 + \frac{p_T^2}{2.8^2} \right)^6}{\left(1 + \frac{p_T^2}{\left(p_1\left(x_F\right)\right)^2} \right)^6} \frac{2.8^2}{\left(p_1\left(x_F\right)\right)^2}\frac{1-\left[ 1+ \frac{\left(P_T^{\textrm{max}}\left(m,x_F\right)\right)^2}{2.8^2}\right]^{-5}}{1-\left[ 1+ \frac{\left(P_T^{\textrm{max}}\left(m,x_F\right)\right)^2}{\left(p_1\left(x_F\right)\right)^2}\right]^{-5}}.
	\label{eq:pT_reWeight}
\end{equation}
The addition of last factor in \cref{eq:pT_reWeight} as compared to previous studies,
is to ensure the normalization is preserved and would not affect the other kinematic
distributions at generator level.


\section{Track Reconstruction}

\section{Event Selection}
The event selection that is used in this study is based on the study by C.~Brown
\cite{chuck-2111} with some modification to increase acceptance at low mass. These cuts
are categorized into different groups and are tabulated in \cref{table:trackCut,table:dimuonCut,table:physCut,table:occCut}.


\begin{table}[ht!]
	\centering
	\caption{Track level cuts.}
	\label{table:trackCut}
	\begin{tabular}{|m{4.5cm}|m{7cm}|m{3cm}|}
		\hline
		Variable                                                                                                                             & Description                                                                          & Cut                          \\ \hline
		$\chi^2_{target}$                                                                                                                    & $\chi^2$ when tack is   forced to pass through center of target ($z=\SI{-129}{\cm}$) & $< 15$                       \\ \hline
		$pz_1$                                                                                                                               & z momentum at station 1                                                              & (\SI{9}{\GeV},\SI{75}{\GeV}) \\ \hline
		nHits                                                                                                                                & total number of hits on wire chambers by each   muon track                           & $> 13$                       \\ \hline
		$x_T^2 +(y_T - \textrm{beamOffset})^2$                                                                                               & radial distance of track from   beam line at the center of target                    & $< \SI{320}{\cm\squared}$    \\ \hline
		$x_D^2 +(y_D - \textrm{beamOffset})^2$                                                                                               &
		radial distance of track from   beam line at the center of beam dump ($z=\SI{42}{\cm}$)                                              &
		(\SI{8}{\cm\squared},\SI{1100}{\cm\squared})   \footnotemark[1]                                                                                                                                                                                            \\ \hline
		\begin{tabular}[c]{@{}c@{}}$\chi^2_{target}<1.5\chi^2_{upstream}$\\      $\chi^2_{target}<1.5\chi^2_{dump}$\end{tabular}             &
		$\chi^2$ when tack is   forced to pass through $z=\SI{-490}{\cm}$(upstream), $z=\SI{-129}{\cm}$(traget) and   $z=\SI{42}{\cm}$(dump) &
		\\ \hline
		$z_0$                                                                                                                                & z position of track's closest approach to beam   line                                & (\SI{320}{\cm},\SI{-5}{\cm}) \\ \hline
		$\chi^2/(\textrm{nHits}-5)$                                                                                                          & $\chi^2/\textrm{ndf}$                                                                & $<12$                        \\ \hline
		$y_1/y_3$                                                                                                                            & y position of track ast St.~1   and St.~3                                            & $<1$                         \\ \hline
		$y_1\times y_3$                                                                                                                      &                                                                                      & $>0$                         \\ \hline
		$| |px_1 - px_3| -0.416|$                                                                                                            & difference in x momentum at St.~1 and St.~3 accounting for the Kmag kick             & $<\SI{0.008}{\GeV}$          \\ \hline
		$|py_1 - py_3|$                                                                                                                      & difference in y momentum at St.~1 and St.~3                                          & $<\SI{0.008}{\GeV}$          \\ \hline
		$|pz_1 - pz_3|$                                                                                                                      & difference in z momentum at St.~1 and St.~3                                          & $<\SI{0.08}{\GeV}$           \\ \hline
		$|py_1 |$                                                                                                                            & absolute value of the y momentum   at St.~1                                          & $>\SI{0.02}{\GeV}$           \\ \hline
	\end{tabular}
\end{table}
\begin{table}[ht!]
	\centering
	\caption{dimuon level cuts.}
	\label{table:dimuonCut}
	\begin{tabular}{|m{4.5cm}|m{7cm}|m{3cm}|}
		\hline
		Variable                                                                                            & Description                                                                              & Cut                            \\ \hline
		$|dx|$                                                                                              & x position of dimuon vertex                                                              & $<\SI{0.25}{\cm}$              \\ \hline
		$|dy-\textrm{beamOffset}|$                                                                          & y position of dimuon vertex                                                              & $<\SI{0.22}{\cm}$              \\ \hline
		$dz$                                                                                                & z position of dimuon vertex                                                              & (\SI{-280}{\cm},\SI{-5}{\cm})  \\ \hline
		$dx^2+(dy-\textrm{beamOffset})^2$                                                                   & radial distance of the dimuon vertex from beam line                                      & $<\SI{0.06}{\cm\squared}$      \\ \hline
		$|dpx|$                                                                                             & absolute value of dimuon x   momentum                                                    & $<\SI{1.8}{\GeV}$              \\ \hline
		$|dpy|$                                                                                             & absolute value of dimuon y   momentum                                                    & $<\SI{2}{\GeV}$                \\ \hline
		$dpx^2+dpy^2$                                                                                       & transverse momentum squared of   dimuon                                                  & $<\SI{5}{\GeV}$                \\ \hline
		dpz                                                                                                 & dimuon z momentum                                                                        & (\SI{38}{\GeV},\SI{116}{\GeV}) \\ \hline
		$|\textrm{tackSeparation}|$                                                                         & distance in z between points   of  closest approach between $\mu^+$   and $\mu^-$ track  & $<\SI{270}{\cm}$               \\ \hline
		$\chi^2_{\textrm{dimuon}}$                                                                                   & $\chi^2$ when both $\mu^+$ and   $\mu^-$ tracks are forced to pass through dimuon vertex & $<18$                          \\ \hline
		$\begin{aligned} |\chi^2_{\textrm{target}}(\mu^+) &+ \chi^2_{\textrm{target}}(\mu^-)\\& -\chi^2_{\textrm{dimuon}}| \end{aligned}$ &                                                                                          & $<2$                           \\ \hline
		$y_3(\mu^+) \times y_3(\mu^-)$                                                                      & y position at St.~3 for both tracks                                                      & $<0$                           \\ \hline
		$\textrm{nHists}(\mu^+)+\textrm{nHists}(\mu^-)$                                                     & sum of the total number if hits on wire chamber by $\mu^+$ and $\mu^-$ track             & $>29$                          \\ \hline
		$\textrm{nHists}_1(\mu^+)+\textrm{nHists}_1(\mu^-)$                                                 & sum of the total number if hits on St.~1 wire chamber by $\mu^+$ and $\mu^-$ track       & $>8$                           \\ \hline
		$|x_1(\mu^+) + x_1(\mu^-)|$                                                                         & sum off x position of of tracks   at St.~1                                               & $>\SI{42}{\cm}$                \\ \hline
	\end{tabular}
\end{table}
\begin{table}[h!]
	\centering
	\caption{ physics cuts}
	\label{table:physCut}
	\begin{tabular}{|m{4.5cm}|m{7cm}|m{3cm}|}
		\hline
		Variable         & Description                          & Cut                                                            \\ \hline
		mass             & dimuon mass                          & (\SI{2}{\GeV},\SI{8.8}{\GeV}) \footnotemark[1]\footnotemark[2] \\ \hline
		$x_F$            & Feynman $x$                            & (-0.1,0.95)                                                    \\ \hline
		$x_{\textrm{target}}$     & Bjorken $x$ of target                  & (0.005,0.55)  \footnotemark[1]                                 \\ \hline
		$|\cos(\theta)|$ & polar angle in Collins-Soper   frame & $<0.5$                                                         \\ \hline
	\end{tabular}
\end{table}
\begin{table}[h!]
	\centering
	\caption{occupancy cuts}
	\label{table:occCut}
	\begin{tabular}{|m{4.5cm}|m{7cm}|m{3cm}|}
		\hline
		Variable                          & Description                                & Cut      \\ \hline
		D1                                & total hits on St.~1 Chamber                & $<400$   \\ \hline
		D2                                & total hits on St.~2 Chamber                & $<400$   \\ \hline
		D3                                & total hits on St.~3 Chamber                & $<400$   \\ \hline
		$D1+D2+D3$                        & total numbers of hits in all wire Chambers & $<1000$  \\ \hline
		Trigger intensity\footnotemark[3] & numbers of proton in the triggering bucket & $<80000$ \\ \hline
	\end{tabular}
\end{table}

\footnotetext[1]{modified from Ref.~\cite{chuck-2111}}
\footnotetext[2]{The 0.99 scaling factor is applied to MC}
\footnotetext[3]{Not applied to MC or Mix background}


\section{Target Contamination}
For a pure target, the cross section can be obtained from the yield using the following
\begin{equation}
	\sigma = \frac{Y A}{T N_A P X \epsilon},
\end{equation}
where $Y$ is the extracted yield, $A$ is the atomic mass of the target,
$T$ is the thickness of the target, $N_A$ is the Avogadro’s number,
$P$ is the proton on target, $\epsilon$ is the acceptance and efficiency correction,
and $X$ is the beam attenuation factor given by
\begin{equation}
	X=\frac{\lambda}{L\rho} \left[1-\exp\left(-\frac{L\rho}{\lambda}\right)\right],
\end{equation}
where $\lambda$ is the interaction length, $L$ and $\rho$ are the length and density
of the target.

The liquid targets are held in a flask \SI{50.8}{\cm} long and \SI{7.62}{\cm} in diameter
and can contain \SI{2.2}{\l} of liquid. The liquid Hydrogen used is \SI{99.999}{\percent}
pure. On the other hand, the Deuterium used came from two sources:
\begin{itemize}
	\item \SI{95.8\pm0.2}{\percent} pure deuterium that was used for bubble chamber experiments
	      at Fermilab, with contamination in the form of \ce{HD}.
	\item \SI{99.99}{\percent} pure commercial deuterium which is used in later part of the experiment.
\end{itemize}
\begin{table}[h!]
	\centering
	\caption{Summary of \ce{LD_2} contamination, taken from Ref.~\cite{don-4993}}
	\label{table:LD2_contamination}
	\begin{tabular}{|l|ll|l|l|}
		\hline
		Sample no. & \multicolumn{2}{l|}{\ce{D_2} bottle no.} & Sample date & description                                                                                                                          \\ \hline
		1          & \multicolumn{1}{l|}{Fermilab}            & 53          & 4/12/18     & \SI{95.6}{\percent} \ce{D}, \SI{4.4}{\percent} \ce{H}; \SI{92}{\percent}  \ce{D_2}, \SI{8}{\percent} \ce{HD} gases     \\
		2          & \multicolumn{1}{l|}{Fermilab}            & 113         & 4/12/18     & \SI{96}{\percent} \ce{D}, \SI{4}{\percent} \ce{H}; \SI{93}{\percent}  \ce{D_2}, \SI{7}{\percent} \ce{HD} gases         \\
		3          & \multicolumn{1}{l|}{Fermilab}            & 53          & 4/12/18     & just air; gas must have leaked                                                                                         \\
		4          & \multicolumn{1}{l|}{Matheson}            & 127         & 4/12/18     & about half air; remaining \SI{99.1}{\percent} \ce{D}, \SI{0.3}{\percent} \ce{H}                                        \\
		5          & \multicolumn{1}{l|}{Matheson}            & 2           & 4/12/18     & sample for test purposes; not analyzed                                                                                 \\
		6          & \multicolumn{1}{l|}{Matheson}            &             & 7/28/16     & more than half air; remaining \SI{99.3}{\percent} \ce{D}, \SI{0.7}{\percent} \ce{H}                                    \\
		7          & \multicolumn{1}{l|}{Matheson}            &             & 5/28/17     & \SI{99.8}{\percent} \ce{D}, \SI{0.2}{\percent} \ce{H}; \SI{99.6}{\percent}  \ce{D_2}, \SI{0.4}{\percent} \ce{HD} gases \\ \hline
	\end{tabular}
\end{table}
Therefore, care is needed to account for the contribution from the hydrogen contamination in deuterium
target data. The result of the mass spectroscopy of the deuterium gas \cite{don-4993} is
summarized in \cref{table:LD2_contamination}. Based this analysis, it is concluded the Fermilab
deuterium contains \SI{91.6}{\percent} \ce{D_2} and \SI{8.4}{\percent} \ce{HD} by moles. \pdfcomment{need to check this sentence.}

The yield of the contaminated hydrogen can be expressed as
\begin{equation}
	Y_{\textrm{cont.~\ce{LD_2}}} = N_A P_D X_D \epsilon_D \left( T_D^D \sigma_{pd}/A_D + T^D_H \sigma_{pp}/A_H   \right).
\end{equation}
Here $T_D^D$ and $T^D_H$ are the thickness of deuterium and hydrogen in the deuterium target.
The subscript $D$ denote these deuterium target specific quantities.

In order to extract the $T_D^D$ and $T^D_H$ from the mole fraction listed in \cref{table:LD2_contamination},
we first note that the volume of a \ce{HD} molecule is about \num{1.094} times the volume of a \ce{D_2}
molecule. Therefore the effective volume of molecules in the target is
\begin{equation}
	\begin{split}
		V_{\text{eff.}}&=\frac{N_{\ce{HD}} V_{\ce{HD}} + N_{\ce{D_2}} V_{\ce{D_2}}}{N_{\text{tot.}}}\\
		&=V_{D_2} \left[ C \frac{V_{\ce{HD}}}{V_{\ce{D_2}}} + (1-C) \right]\\
		&=V_{D_2} f,
	\end{split}
\end{equation}
where $f=\left[ C \frac{V_{\ce{HD}}}{V_{\ce{D_2}}} + (1-C) \right]$.
The total number of molecules per area is
\begin{equation}
	\begin{split}
		\frac{N_{\textrm{tot.}}}{\textrm{Area}} &= \frac{L}{V_{eff}}\\
		&= \frac{L}{V_{D_2}f}=\frac{L\rho_{D_2}}{2A_Df}.
	\end{split}
\end{equation}
The thickness of D (in \unit{\g\per\cm\squared}) in the target cell is
\begin{equation}
	\begin{split}
		T_D^D &= \frac{N_D A_D}{A} = A_D\frac{N_{\textrm{tot.}}}{\textrm{Area}} \left[ 2(1-C) + C \right]\\
		&=A_D \frac{L\rho_{\ce{D_2}}}{2A_Df}(2-C)\\
		&=\frac{L\rho_{D_2}}{f}(1-C/2).
	\end{split}
\end{equation}
The thickness of H (in \unit{\g\per\cm\squared}) in the target cell is
\begin{equation}
	\begin{split}
		T^D_H &= \frac{N_H A_H}{A} = A_H\frac{N_{\textrm{tot.}}}{\textrm{Area}}C\\
		&=\frac{L\rho_{\ce{D_2}}}{f}\frac{A_H}{A_D}\frac{C}{2}.
	\end{split}
	\label{eq:TDH}
\end{equation}
To determine the density of the deuterium target, the vent pressure is measured and is compared
with NIST database to obtained the expected density for pure deuterium ($\rho_{\ce{D_2}}$) \cite{density-1453}.
For contaminated deuterium, the effective density is
\begin{equation}
	\begin{split}
		\rho_{\textrm{eff.}} &= \frac{1}{L} (T_D^H + T_D^D)\\
		&=\frac{\rho_{\ce{D_2}}}{f} \left[ \frac{A_H}{A_D}\frac{C}{2}+(1-C/2) \right].
	\end{split}
\end{equation}
and the effective interaction length is
\begin{equation}
	\begin{split}
		\frac{1}{\lambda_{\textrm{eff.}}} &= \frac{1}{L\rho_{\textrm{eff.}}} \left[\frac{T_D^H}{\lambda_H} +\frac{T_D^D}{\lambda_D}\right]\\
		&=\left[\frac{A_H}{A_D}\frac{C}{2\lambda_H} + \frac{1-C/2}{\lambda_D}\right]\left( \frac{A_H}{A_D}\frac{C}{2} +(1-C/2)\right)^{-1}.
	\end{split}
\end{equation}

Note that in previous studies and publication, including Ref.~\cite{dove2021}, the ratio of the atomic mass
is missing in \cref{eq:TDH}. This causes a roughly \SI{2}{\percent} difference in the cross section ratio
that will be discussed later.

Thus the deuterium cross section is
\begin{equation}
	\sigma_{pd} = \frac{Y_{\ce{LD_2}} A_D}{T_D^D N_A P_D X_{\text{eff}} \epsilon_D} - \frac{T^D_H}{T_D^D} \frac{A_D}{A_H} \sigma_{pp},
\end{equation}
where
\begin{equation}
	\sigma_{pp} = \frac{Y_{\ce{LH_2}} A_H}{T_H^H N_A P_D X_{H} \epsilon_H}.
\end{equation}
And the cross section ratio is given by
\begin{equation}
	\frac{\sigma_{pd}}{2\sigma_{pp}} = \frac{Y_{\ce{LD_2}}}{2Y_{\ce{LH2}}}\frac{A_D}{A_H}\frac{T_H^H P_H X_{H} \epsilon_H}{T_D^D P_D X_{\textrm{eff}} \epsilon_D} - \frac{T^D_H}{2T_D^D} \frac{A_D}{A_H}.
\end{equation}

The switch to commercial pure deuterium gas happened during Roadset 67 data taking. Therefore,
an average contamination, weighted by the proton on target, is used for the entire dataset.
The average contamination for entire roadset 57 -70 data is determined to be $C=\SI{5.95}{\percent}$.
Therefore the effective values for target contamination for this dataset are
\begin{equation}
	\begin{split}
		T_D^H &= \SI{0.1230}{\g\per\cm\squared}, \\
		T_D^D &= \SI{8.009}{\g\per\cm\squared},\\
		\rho_{\textrm{eff.}} & = \SI{0.1601}{\g\per\cm\squared},\\
		\lambda_{\textrm{eff.}} &= \SI{71.39}{\g\per\cm\squared},\\
		\lambda_{\textrm{eff.}}/\rho_{\textrm{eff.}} &= \SI{446.0}{\cm},\\
		X_{\textrm{eff.}} &= 0.9451.
	\end{split}
\end{equation}
And since there is no contamination in the \ce{LH_2} target
\begin{equation}
	\begin{split}
		\rho_{H_2} &= \SI{0.0708}{\g\per\cm\cubed},\\
		T_H^H &= L\rho_{\ce{H_2}} = \SI{3.597}{\g\per\cm\squared},\\
		\lambda_H/\rho_{\ce{H_2}} &= \SI{734.5}{\cm},\\
		X_H &=0.9662.
	\end{split}
\end{equation}

\section{Intensity Extrapolation}
\label{sec:extrapolation}
Extracting the Drell-Yan cross section ratio using the intensity extrapolation method
is first studied by J.~Dove \cite{dove2020} and A.~Tadepalli \cite{tadepalli2019}.
The idea behind this method is that the number of observed Drell-Yan
events ,in the absent of any spectrometer effects, should be linearly proportional to the beam
intensity. On the contrary, the two muons in the accidental background are typical coming from
two separate interactions, and there it is expected proportional to the intensity squared. By taking the ratio of the
$(p+p)$ and $(p+d)$ dimuon event rates, other rate dependence effect would also be cancel out
at zero intensity.

The procedure for this method is summarized here. The dimuon selection is applied to remove contributions
from the charmonium decay and to exclude region close to the boundaries of the acceptance.
Then the background originated from the interaction with the instruments is estimated by using
the empty flask data normalized by the integrated beam intensity, and is subtracted from the
$p+p$ and $p+d$ data. Third, the $(p+d)/2(p+p)$ ratios are formed by empty flask subtracted
dimuon data, normalized by the beam intensity. The ratio is calculated as a function of the
instantaneous beam intensity measured by the BIM. The ratio is then fitted with the following function
\begin{equation}
	R_i \left(I\right) = p_{0i} + p_1 I + p_2 I^2,
\end{equation}
where the parameter $p_1$ and $p_2$ are common to all bins, and the intercept $p_{0i}$ give
the value of the cross section ratio in each $x_T$ bins.

\section{Mass Spectrum Fitting}
\pdfmargincomment{see analysis note}

\subsection{TFractionFitter}
To account for the statistical uncertainties in both the data and Monte Carlo
simulation, the TFrationFitter algorithm is used in this analysis \cite{barlow1993}.
This is achieved by performing a standard likelihood fit using Poisson statistics,
while the template, generated from MC, are also varied within statistics, leading
to additional contributions to the overall likelihood.

Let there are $m$ sources. The number of MC events in bin $i$ from the $j$ source
is given by $a_{ji}$. For each source, there is some (unknown) expected distribution
$A_{ji}$. The expected number of events in each bin is given by
\begin{equation}
	f_i = \sum^m_{j=1} p_j A_{ji}.
	\label{eq:TF_f}
\end{equation}
The total likelihood is the combined probability of the observed $\left\{d_i\right\}$
and the observed $\left\{a_{ji}\right\}$
\begin{equation}
	\ln \mathcal{L} = \sum^n_{i=1} d_i \ln f_i -f_i + \sum^n_{i=1} \sum^m_{j=1} a_{ji} \ln A_{ji} - A_{ji}.
	\label{eq:TF_likelihood}
\end{equation}
The estimates for the strength $p_j$ and the expected distribution $A_{ji}$ are
found by maximizing this likelihood. One consequence of this approach is the
$n \cross m$ fit parameters $A_{ji}$. However, the  minimization of these additional
parameters is done analytically ratter than treating them as formal fit parameters.

In the case of weighted MC, \cref{eq:TF_f} is modified into
\begin{equation}
	f_i = \sum^m_{j=1} p_j w_{ji}A_{ji},
\end{equation}
where $w_{ji}$ is the average weight of the MC events in bin $i$ from the $j$th source.

\section{Acceptance Calculation}

\section{Tracker Efficiency}

\section{Drell-Yan NLO calculation}
The next-to-leading order (NLO) calculation is done using a parton level Monte
Carlo program written by the INFN group\cite{catani2009,catani2007}.

\ifSubfilesClassLoaded{ \printbibliography[heading=bibintoc,title={References}]}{}
\end{document}
