\documentclass[../main.tex]{subfiles}
\begin{document}

\ifSubfilesClassLoaded{
	\mainmatter
	\setcounter{chapter}{5}
}{}

\chapter{Conclusion and Future Prospects}
\label{ch:conclusion}

In addition to the results presented, there is ongoing analysis on extracting 
the angular distribution of the Drell-Yan process as well as the charmonium 
production from the SeaQuest data. The analysis of the angular distribution would 
utilize the data from the dump, where most of the proton beam interacts.

\pdfmargincomment{nuclear dependence of DY and charmonium}

SpinQuest is expected to run until summer 2024\pdfcomment{check date}. The Sivers
asymmetry in both Drell-Yan and charmonium production can be extracted from the full
data set once it is made available.
\pdfmargincomment{full datasets from spinquest}

There is also interest in using the same spectormeter for dark photon search
\cite{apyan2022}. An electromagnetic calorimeter would be placed behind the 
Station 3 chambers and in front of the absorber for electron identification.
\pdfmargincomment{interest in searching for dark photon using the same spetrometer}

\ifSubfilesClassLoaded{ \printbibliography[heading=bibintoc,title={References}]}{}
\end{document}
