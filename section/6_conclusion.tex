\documentclass[../main.tex]{subfiles}
\begin{document}

\ifSubfilesClassLoaded{
	\mainmatter
	\setcounter{chapter}{5}
}{}

\chapter{Conclusion and Future Prospects}
\label{ch:conclusion}
Since its inception, Drell-Yan process has provided invaluable information
on the internal structure of hadrons.
As electromagnetic processes with well understood reaction mechanisms,
the Drell-Yan process, as well as DIS, have provided important constrains to the PDFs. 
The SeaQuest experiment allows for simultaneous measurements of the Drell-Yan process
and charmonium production in $p+p$ and $p+d$ interactions at \SI{120}{\GeV}.
The charmonium production data is also sensitive to the gluon content of the nucleon.

In this thesis, the Drell-Yan $\sigma_{pd}/2\sigma_{pp}$ cross section ratio from the full
SeaQuest data set is presented. This ratio is particularly useful in constraining
the light sea-quark asymmetry in the proton. The SeaQuest measurement extends the
previous measurements to $x$ of $0.45$.
The data has been analyzed using two independent methods.
First, the background events are modeled using a combination of simulation and
track mixing. The yields for different processes can be extracted from the fit to
the mass spectrum.
Second, exploiting the difference in the intensity dependence of the signal and background events,
the cross section ratio can be obtained by extrapolating down to zero intensity.

Unlike previous measurements by E866, the  Drell-Yan $\sigma_{pd}/2\sigma_{pp}$ ratio
is consistently above unity across the measured kinematics.
This suggests the $\bar{d}>\bar{u}$ for up to $x$ of $0.45$.
The first result from SeaQuest has been included in various global PDF analysis
and the result from the full data set has reduced the statistical uncertainties.
The extracted cross section ratio from the Run~2-3
data are consistent between the two different methods. However, the massfit
and intensity extrapolation methods disagree on the Run~5-6 result,
suggesting a larger systematic uncertainties than previous estimation.
In particular, the systematic uncertainties in the intensity extrapolation
method is larger than the massfit method, primarily dominated in the difference
between the fit functions.

The measurements of the absolute charmonium production cross section from $p+p$
and $p+d$ at $\sqrt{s}=\SI{15.06}{\GeV}$ are also reported. The charmonium measurements
are sensitive to both the quark and gluon distributions in the nucleon. The two datasets
are analyzed separately and the extracted cross sections are very consistent between datasets.
The measured $J/\psi$ and $\psi'$ $x_F$ distributions are in good agreement with the NRQCD predictions.
The measured $\sigma_{\psi'}/\sigma_{J/\psi}$ ratios are found to increase as $x_F$ increases.
This suggests the $x_F$ distribution for $\psi'$ is wider than $J/\psi$.
This also indicates that quark-antiquark annihilation is more important in $\psi'$ production than in $J/\psi$.
This behavior is well described by the NRQCD calculations.
The $P_T$ distributions for $J/\psi$ and $\psi'$ are also reported and compared to experiments over a wide range of $\sqrt{s}$.
It is found that the $\expval{P_T^2}$ increases logarithmically versus $\sqrt{s}$ over a wide range of energy.
The charmonium measurements on $p+p$ and $p+d$ would also form the basis of
extracting the nuclear dependence in the charmonium production.

A direct comparison of $\sigma_{pd}/2\sigma_{pp}$
between $J/\psi$ production and the Drell-Yan process is also presented. 
While the Drell-Yan process proceeds via $q \bar{q}$ annihilation,
$J/\psi$ production has contributions from both the $q \bar{q}$ annihilation and the gluon fusion.
The measured $\sigma_{pd}/2\sigma_{pp}$ ratios are greater than unity for both the Drell-Yan and $J/\psi$ production,
showing that both processes are sensitive to the $\bar{d},\,\bar{u}$ flavor asymmetry of the proton sea.
The smaller values of $\sigma_{pd}/2\sigma_{pp}$ for $J/\psi$
production reflect the dilution due to the additional contribution
of gluon fusion for charmonium production. 
It would be interesting to include the $\sigma_{pd}/2\sigma_{pp}$ $J/\psi$ data in a future
extraction of the $\bar{d}/ \bar{u}$ asymmetry of the proton. 

In addition to the results presented, there is ongoing analysis on extracting
the angular distribution of the Drell-Yan process as well as the charmonium
production from the SeaQuest data. The analysis of the angular distribution would
utilize the data from the dump, where most of the proton beam interacts.

SpinQuest \cite{geesaman2014}, the successor to the SeaQuest, will utilize the same beamline and
spectrometer to measure the transverse momentum dependent parton distribution
function of the nucleon. The target would be replaced by transversely polarized \ce{NH_3}
and \ce{ND_3} targets~\cite{crabb1995}. The goal is to measure the azimuthal asymmetry in
Drell-Yan process and $J/\psi$ production, which can provide information on the
transverse motion of the partons in the nucleons.
SpinQuest is expected to run until the long shutdown starting summer 2025. With the experience
from SeaQuest analysis, SpinQuest should allocate more DAQ bandwidth in collecting
like-sign and/or single track events. This would allow better modeling of the combinatorial
background. Given that SpinQuest would utilize a polarized target,
the average beam intensity would likely be lower than that was delivered to SeaQuest.
%However, due to the design of Main Injector,
%the beam quality might be lower at lower intensity, resulting in larger intensity dependent effects.

There is also interest in using the same spectrometer for dark photon search~\cite{apyan2022}.
The aim is to detect dilepton pairs with vertex downstream of FMag.
Two fiber scintillator hodoscopes, with better spacial resolution than the existing hodoscopes,
have already been installed on both sides of KMag,
which would provide better vertex resolution at the trigger level.
There is also a plan to install an electromagnetic calorimeter behind the
Station 3 chambers for electron identification, either in front of the absorber or replacing the absorber.
This would allow detecting dark photon in the electron channel.

While the Drell-Yan measurements at SeaQuest have provided strong constraints on the light sea-quark
asymmetry at large $x$, the sea quark at small $x$ remains poorly understood.
Semi-inclusive DIS measurements at Jefferson Lab and the future Electron-Ion Collider~\cite{abdulkhalek2022}
would provide important constraints on the small $x$ behavior. 
The electro- and photo-production of $J/\psi$ at Jefferson Lab and EIC would also provide important
constraints on the LDMEs~\cite{qiu2021}, and test the NRQCD factorization as well as the universality of the LDMEs. 
There are also interest in extending the study of sea-quark to other flavors. 
With the recent results from LHCb~\cite{aaij2022} and NNPDF\cite{ball2022},
there is a lot of interest in investigate the charm sea in the nucleon~\cite{vogt2021,vogt2023}.
The charmonium production would be an important tool for probing the intrinsic charm in the proton.

\ifSubfilesClassLoaded{ \printbibliography[heading=bibintoc,title={References}]}{}
\end{document}
