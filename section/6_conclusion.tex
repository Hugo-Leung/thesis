\documentclass[../main.tex]{subfiles}
\begin{document}

\ifSubfilesClassLoaded{
	\mainmatter
	\setcounter{chapter}{5}
}{}

\chapter{Conclusion and Future Prospects}
\label{ch:conclusion}
Since its proposal, the Drell-Yan measurements has provided invaluable information
on the internal structure of hadrons. As the long distance interaction within a
hadron could not be calculated with perturbative techniques, Drell-Yan measurements,
as wll as DIS data, provide important constrains to the parton distribution functions
(PDFs). Improvements in PDFs are important in modern precision measurements, as
demonstrated in the recent W mass measurement by CDF.

In this thesis, the Drell-Yan $(p+d)/2(p+p)$ cross section ratio from the full
SeaQuest data set are presented. This ratio is particularly useful in constraining
the light sea-quark asymmetry in the proton. The SeaQuest measurement extends the
previous measurements to $x$ of $0.45$.
The data has been analyzed using two independent methods. First, exploiting the
difference in the intensity dependence of the signal and background events, the
cross section ratio can be obtained by extrapolating down to zero intensity.
Second, the background events are modeled using a combination of simulation and
track mixing. The yield for different process can be extracted from the fit to
the mass spectrum.

Unlike previous measurements by E866, the  Drell-Yan $(p+p)/2(p+p)$ ratio
stays above unity across the measured kinematics. This suggests the $\bar{d}>\bar{u}$
for up to $x$ of $0.45$.
The first result from SeaQuest has been included in various global PDF analysis
and the result from the full data set would be important in understanding the
experimental uncertainties. The extracted cross section ratio from the Run 2-3
data are consistent between the two different methods. However, the massfit
and intensity extrapolation methods disagree on the Run 5-6 result,
suggesting a larger systematic uncertainties than previous estimation.

The measurements of the absolute charmonium production cross section from $p+p$
and $p+d$ at $\sqrt{s}=\SI{15.06}{\GeV}$ are also reported. The charmonium measurements
are also sensitive to the gluon distribution in the nucleon. The two datasets
are analyzed separately and the extracted cross section are very consistent between
datasets. The measured
$J/\psi$ $x_F$ distributions are in very good agreement with the NRQCD predictions.
The $x_F$ distribution for the $\psi'$ production is considerably wider than that of the
$J/\psi$, this is likely due to the larger importance of the $q\bar{q}$ annihilation
in $\psi'$. Although the systematic uncertainties are large, the measured $\psi'$
$x_F$ distribution is also wider than the NRQCD calculations. It is possible that
the LDMEs used are for the $\psi'$ are yet to be fully optimized, and this new
data can also be used to constrain the LDMEs.
The charmonium measurements on $p+p$ and $p+d$ are would also form the basis to
extracted the nuclear dependence in the charmonium production.
\pdfmargincomment{nuclear dependence of DY and charmonium}

In addition to the results presented, there is ongoing analysis on extracting
the angular distribution of the Drell-Yan process as well as the charmonium
production from the SeaQuest data. The analysis of the angular distribution would
utilize the data from the dump, where most of the proton beam interacts.

SpinQuest \cite{geesaman2014}, the successor to the SeaQuest, will utilized the same beamline and
spectrometer to measured the transverse momentum dependent parton distribution
function of the nucleon. The target would be replaced by transversely polarized \ce{NH_3}
and \ce{ND_3} targets. The goal is to measure the azimuthal asymmetry in
Drell-Yan process and $J/\Psi$ production, which can provide information on the
the transverse motion of the partons in the nucleons.
SpinQuest is expected to run until summer 2025\pdfcomment{check date}. With the experience
from SeaQuest analysis, SpinQuest should allocate more DAQ bandwidth in collecting 
like-sign and/or single track events. This would allow better modeling of the combinatorial 
background. Given SpinQuest would utilize a polarized target,
the average beam intensity would like be lower than that was delivered to SeaQuest.
However, due to the nature of resonant extraction at Main Injector,
the beam quality might be lower at lower intensity, resulting in larger intensity dependent effects.

There is also interest in using the same spectrometer for dark photon search~\cite{apyan2022}. 
The aim is to detect dilepton pairs with vertex downstream of FMag.
Two fiber scintillator hodoscopes, with better spacial resolution, have already installed
on either side of KMag, which would provide better vertex resolution.
There is also a plan to installed an electromagnetic calorimeter behind the
Station 3 chambers for electron identification, either in front of the absorber or replacing the absorber.
This would allow detecting dark photon in the electron
channel.

\ifSubfilesClassLoaded{ \printbibliography[heading=bibintoc,title={References}]}{}
\end{document}
