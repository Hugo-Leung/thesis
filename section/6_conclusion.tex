\documentclass[../main.tex]{subfiles}
\begin{document}

\ifSubfilesClassLoaded{
	\mainmatter
	\setcounter{chapter}{5}
}{}

\chapter{Conclusion and Future Prospects}
\label{ch:conclusion}
In this thesis, the analysis on the dimuon production data in $p+p$ and $p+d$ interactions 
at \SI{120}{\GeV} performed in the SeaQuest experiment at Fermilab is presented.
The $\sigma_{pd}/2\sigma_{pp}$ Drell-Yan cross section ratio
is particularly sensitive to the light sea-quark asymmetry in the proton. 
This new measurement extends the previous measurements to $x$ of $0.45$.
The charmonium production data, which is sensitive to both antiquark and gluon
distributions in the nucleons, has also been analyzed.

For the Drell-Yan part of the analysis, two independent analysis methods have been adopted.
We found that the Drell-Yan $\sigma_{pd}/2\sigma_{pp}$ ratio is consistently above unity, which
also shows that the $\bar{d}/\bar{u}$ ratio continue to increase with $x$ between $0.13<x<0.45$.
While the SeaQuest result is in qualitative agreement with those of E866 at small $x$,
some tension in the $\bar{d}/\bar{u}$ ratios at large $x$ remains.
The new results of $\bar{d}/\bar{u}$ support various theoretical models, including meson cloud
and statistical models, which predict these ratios should continue to rise as $x$ increases.
The first result from SeaQuest has been included in various global PDF analysis.
These new proton PDFs have shown that the SeaQuest data significantly reduce the uncertainties
of the flavor asymmetry at large $x$.
With the reduced statistical uncertainties form the analysis of the full data set presented,
the antiquark distributions in the proton can be further constrained in future global analyses.
These results would also have implications on the reach of future collider experiments for new physics~\cite{amoroso2023}.
For example, in the search of $W'/Z'$ particles at mass scales of \SIrange{4}{5}{\TeV},
the bounds are currently limited by the PDF uncertainties at large $x$~\cite{brady2012}.
Measurements such as the Drell-Yan results presented here would be important for constraining the PDFs in this region.

Using similar analysis method, the charmonium production cross sections from $p+p$ and $p+d$ are also extracted.
%The measured $J/\psi$ and $\psi'$ $x_F$ distributions are in good agreement with the NRQCD predictions.
The $x_F$ distribution for $\psi'$ is found to be wider than $J/\psi$, as indicated by the $\sigma_{\psi'}/\sigma_{J/\psi}$ ratio.
This shows that $q\bar{q}$ annihilation is more important in $\psi'$ production than in $J/\psi$,
as the $q\bar{q}$ annihilation would give a broader $x_F$ distribution than the gluon fusion process.
This interpretation is well supported by the NRQCD calculations.
%The $P_T$ distributions for $J/\psi$ and $\psi'$ are also obtained and compared with results from other experiments over a wide range of $\sqrt{s}$.
%It is found that the $\expval{P_T^2}$ increases logarithmically with $\sqrt{s}$ over a wide range of energy.

A direct comparison of $\sigma_{pd}/2\sigma_{pp}$
between $J/\psi$ production and the Drell-Yan process is also presented. 
While the Drell-Yan process proceeds via $q \bar{q}$ annihilation,
$J/\psi$ production has contributions from both the $q \bar{q}$ annihilation and the gluon fusion.
The measured $\sigma_{pd}/2\sigma_{pp}$ ratios are greater than unity for both the Drell-Yan and $J/\psi$ production,
showing that both processes are sensitive to the $\bar{d},\,\bar{u}$ flavor asymmetry of the proton sea.
The smaller values of $\sigma_{pd}/2\sigma_{pp}$ for $J/\psi$
production reflect the dilution due to the additional contribution
of gluon fusion for charmonium production. 
It would be interesting to include the $\sigma_{pd}/2\sigma_{pp}$ $J/\psi$ data in a future
extraction of the $\bar{d}/ \bar{u}$ asymmetry of the proton. 

Following the success of SeaQuest, SpinQuest will measure the transverse momentum dependent parton distribution
functions (TMDs) of the nucleon by replacing the SeaQuest target with transversely polarized \ce{NH_3}
and \ce{ND_3} targets~\cite{geesaman2014}.
By utilizing proton induced Drell-Yan, SpinQuest would be uniquely positioned to
measure the flavor dependence of the TMDs in the antiquark region.
%SpinQuest is expected to run until the long shutdown starting summer 2025. 
With the experience from SeaQuest analysis, SpinQuest should allocate more DAQ bandwidth in collecting
like-sign and/or single track events. This would allow better modeling of the combinatorial
background.

While the Drell-Yan process is used to probe the large $x$ behavior, 
the small $x$ region, where abundant sea-quark reside, can be better probed using
semi-inclusive DIS measurements at Jefferson Lab and the future Electron-Ion Collider~\cite{abdulkhalek2022}.
The electro- and photo-production of $J/\psi$ at Jefferson Lab and EIC would also provide important
constraints on the LDMEs, and test the NRQCD factorization as well as the universality of the LDMEs~\cite{qiu2021}. 
There are also interest in extending the study of sea-quark to other flavors. 
With the recent results from LHCb~\cite{aaij2022} and NNPDF\cite{ball2022},
there is resurgence of interest in investigating the charm sea in the nucleon~\cite{chang2011,chang2011a}.
The charmonium production would be an important tool for probing the intrinsic charm in the proton~\cite{vogt2021,vogt2023}.

\ifSubfilesClassLoaded{ \printbibliography[heading=bibintoc,title={References}]}{}
\end{document}
