\documentclass[../main.tex]{subfiles}
\begin{document}

\begin{dedication}
	In loving memory of my grandmother, who could not see the completion of this dissertation.
\end{dedication}

\begin{acknowledgments}
	This thesis is a product of collaboration with a large number of talented
	individuals with whom I have had the privilege of working with over the past
	few years. I would like to take this opportunity to thank all of them for all
	their support, without which it would not be possible to finish this work.

	First and foremost I would like to thank my advisor, Prof.~Jen-Chieh Peng, for
	his continuous support and assistance. His constant encouragement and feedback
	have be extremely important at every step in this research. 
	I would also like to express my	appreciation for his efforts and guidance
	through out this whole journey, which helped me learn and develop a broad
	understanding of the field. 
	Without his constant encouragement, support
	and reassurance, I would not  make it through this whole journey.
	I must also express my apologies for delaying his post-retirement plans.

	I would also like to thank former graduate students Jason Dove and Shivangi
	Prasad with helping me learn the ropes. Discussions with both Jason and
	Shivangi have been very helpful in organize my thoughts.
	I must thank Shivangi for her continued effort in making herself available for 
	discussions after moving on the new projects at Argonne National Lab, which
	without doubt would require much of her time and energy.

	I would like to thank both the SeaQuest and SpinQuest collaborations.
	The data I have analyzed were collected before I joined SeaQuest, as such,
	I would like to thank everyone who participated in the construction, maintenance,
	and data taking of the experiment.
	This study would not be possible with out the tremendous work by all the collaborators. 
	I would like to thank Dr.~Kun Liu for teaching me about the DAQ system.
	I would also like to thank of Dr.~Kenichi Nakano for his help in my analysis.
	His attention to detail has been extremely helpful in verifying every step
	of my work.
	I would like to thank Dr.~Rick Tesarek for all his help and guidance during my
	stay at Fermilab. 
	I would also need to thank Dr.~Steve Timm for all his help navigating Fermilab 
	computing policy and his assistance in maintaining our computing system.
	

	Last and certainly not least, I would like to extend my gratitude to my family
	for their support in my pursuit of a Ph.D halfway across the glob, particularly
	during the pandemic. While this whole journey have been enjoyable and rewarding, 
	I do regret not being able spend more time at home.

	This research is supported by National Science Foundation via grant NSF-PHYS2111046.

\end{acknowledgments}

\end{document}
