\documentclass[../main.tex]{subfiles}
\begin{document}


\chapter{Definition of Kinematic Variables}
\label{a_ch:kinematic}
In Drell-Yan measurements, momentum fractions$x_1$ and $x_2$ carried by the partons
cannot be measured directly, instead, they are inferred from the
four-momentum of the dimuons. At leading order, the dimuon pair would have little
transverse momentum. However, at higher order, the gluon emission from the quarks
can produce large transverse momentum in the final dimuons. Therefore, different
definitions had been proposed to account for this effect.

To compare the different definitions, we define the following variables with
components given in the hadron-hadron CM frame.
\begin{table}[h!]
	\centering
	\begin{tabular}{l|l}
		variables                                                 & meaning                        \\ \hline
		$P_A = \left(   \sqrt{s}/2, \vec{0}, \sqrt{s}/2 \right)$  & four momentum of beam hadron   \\
		$P_B = \left( \sqrt{s}/2, \vec{0},   -\sqrt{s}/2 \right)$ & four momentum of target hadron \\
		$P_A+P_B=\left(\sqrt{s},\vec{0},0\right)$                 &                                \\
		$q = \left(q_0, \vec{q_T}, q_z\right)$                    & four momentum of the dimuon    \\
		$Q^2=q\cdot q=q_0^2-q_T^2-q_z^2$                          & dimuon mass                    \\
		$y=\frac{1}{2}\ln \frac{q_0 +   q_z}{q_0-q_z}$            & rapidity of the dimuon         \\
		$m_T^2 = q_0^2 - q_z^2=Q^2+q_T^2$                         & transverse mass of the dimuon  \\ \hline
	\end{tabular}
\end{table}

Ignoring the transverse momentum of the virtual momentum, we can define the following
\begin{equation}
	\begin{split}
		x_1-x_2=x_F &= \frac{2q_z}{\sqrt{s}},\\
		x_1x_2 &= \frac{Q^2}{s}.\\
	\end{split}
\end{equation}
Therefore
\begin{equation}
	\begin{split}
		x_1 &= \sqrt{\frac{x_F^2}{4}+Q^2/s}+\frac{x_F}{2} \\
		&=\frac{1}{\sqrt{s}}\left(\sqrt{q_z^2+Q^2}+q_z \right),\\
		x_2 &= \sqrt{\frac{x_F^2}{4}+Q^2/s}-\frac{x_F}{2} \\
		&=\frac{1}{\sqrt{s}}\left(\sqrt{q_z^2+Q^2}-q_z \right).\\
	\end{split}
\end{equation}
This definition has been used in previous fixed target experiments.


Instead of the longitudinal momentum of the virtual photon, one can define
the momentum fractions using the rapidity of the virtual photon, which has the benefit
of only using frame independent quantities.
\begin{equation}
	\begin{split}
		x_1 &= \frac{Q}{\sqrt{s}} e^{+y}\\
		&= \frac{Q}{\sqrt{s}} \sqrt{ \frac{q_0+q_z}{q_0-q_z} }\\
		&= \frac{Q}{\sqrt{s}} \frac{q_0+q_z}{\sqrt{q_0^2-q_z^2}}\\
		&=\frac{Q}{\sqrt{s}m_T}\left(\sqrt{Q^2+q_z^2+q_T^2}+q_z\right),\\
		x_2 &= \frac{Q}{\sqrt{s}} e^{-y}\\
		&= \frac{Q}{\sqrt{s}m_T}\left(\sqrt{Q^2+q_z^2+q_T^2}-q_z\right).\\
	\end{split}
\end{equation}
Therefore
\begin{equation}
	\begin{split}
		x_1x_2 &= \frac{Q^2}{s},\\
		x_1-x_2 &= \frac{2Qq_z}{\sqrt{s}m_T}.
	\end{split}
\end{equation}

Instead of the mass of the virtual photon, one can also use the transverse mass.
\begin{equation}
	\begin{split}
		x_1 &= \frac{m_T}{\sqrt{s}} e^{+y}\\
		&=\frac{1}{\sqrt{s}}\left(\sqrt{Q^2+q_z^2+q_T^2}+q_z\right),\\
		x_2 &= \frac{m_T}{\sqrt{s}} e^{-y}\\
		&= \frac{1}{\sqrt{s}}\left(\sqrt{Q^2+q_z^2+q_T^2}-q_z\right).\\
	\end{split}
	\label{a_eq:def3}
\end{equation}
Therefore
\begin{equation}
	\begin{split}
		x_1x_2 &= \frac{m_T^2}{s} = \frac{Q^2+q_T^2}{s},\\
		x_1-x_2 &= \frac{2q_z}{\sqrt{s}}.
	\end{split}
\end{equation}


At SeaQuest, we defined the momentum fraction based on the ratios of two inner products.
\begin{equation}
	\begin{split}
		x_1 &= \frac{P_B\cdot q}{P_B\cdot (P_A+P_B)}\\
		&= \frac{\sqrt{s}(q_0+q_z)/2}{s/2}\\
		&= \frac{q_0+q_z}{\sqrt{s}}\\
		&= \frac{1}{\sqrt{s}}\left(\sqrt{Q^2+q_z^2+q_T^2}+q_z\right),\\
		x_2 &= \frac{P_A\cdot q}{P_A\cdot (P_A+P_B)}\\
		&= \frac{1}{\sqrt{s}}\left(\sqrt{Q^2+q_z^2+q_T^2}-q_z\right).\\
	\end{split}
\end{equation}
This definition is therefore identical to \cref{a_eq:def3}.

Another definition is
\begin{equation}
	\begin{split}
		x_1 &= \frac{Q^2}{2P_A \cdot q}\\
		&= \frac{Q^2}{\sqrt{s}(q_0-q_z)}\\
		&= \frac{1}{\sqrt{s}}\frac{q_0^2 -q_z^2 -q_T^2}{q_0-q_z} \\
		&= \frac{1}{\sqrt{s}} \left(q_0+q_z -\frac{q_T^2}{q_0-q_z}\right)\\
		&= \frac{1}{\sqrt{s}}\left(\sqrt{Q^2+q_z^2+q_T^2}+q_z -\frac{q_T^2}{q_0-q_z}\right),\\
		x_2 &= \frac{Q^2}{2P_B \cdot q}\\
		&= \frac{1}{\sqrt{s}}\left(\sqrt{Q^2+q_z^2+q_T^2}-q_z -\frac{q_T^2}{q_0+q_z}\right).\\
	\end{split}
\end{equation}
Therefore
\begin{equation}
	\begin{split}
		x_1x_2 &= \frac{Q^4}{s m_T^2}, \\
		x_1-x_2 &= \frac{2q_z}{\sqrt{s}}.
	\end{split}
\end{equation}

These different definitions are consistent when in the limit of dimuon transverse momentum approaching zero.
The difference is in how the higher order correction, and therefore the transverse momentum,
is being handled.

\end{document}
