The cross section for the annihilation process is given by the sum of the LO Drell-Yan expression and
the NLO correction~\cite{kubar1980}
\begin{equation}
	\frac{d^2\sigma^A}{dQ^2dx_{F}} = \sum_q e^2_q \int^1_{x_1} \dd{t_1} \int^1_{x_2} \dd{t_2}
	\left[ \frac{d^2\hat{\sigma}^{DY}}{dQ^2dx_F}+\frac{d^2\hat{\sigma}^{A}}{dQ^2dx_F} \right]
	\left[f_{q/A}\left(t_1\right)f_{\bar{q}/B}\left(t_2\right) +
	f_{\bar{q}/A}\left(t_1\right)f_{q/B}\left(t_2\right)
	\right].
\end{equation}
Where Feynman-$x$ ($x_F$) is typically given by the difference of $x_1$ and $x_2$ and is related to the
longitudinal momentum of the dimuon pair ($P_L$) in the hadron-hadron center-of-mass frame.
\begin{equation}
	x_F = x_1 - x_2 = \frac{2P_L}{\sqrt{s}}.
\end{equation}
The LO Drell-Yan term is given by
\begin{equation}
	\frac{ d^2\hat{\sigma}^{DY} }{dQ^2 dx_F} = \frac{4\pi\alpha^2}{9Q^2 s} \frac{1}{x_1+x_2}\delta\left(t_1-x_1\right)\delta\left(t_2-x_2\right).
\end{equation}
The NLO correction to the annihilation process, from \cref{subfig:DY_gb,subfig:DY_interfer},
is given by
\begin{equation}
\begin{split}
	\frac{d^2\hat{\sigma}^{A}}{dQ^2dx_F} =& \frac{1}{2}A \frac{\delta\left(t_1-x_1\right)\delta\left(t_2-x_2\right)}{\left(x_1+x_2\right)} \left[ 1+\frac{5}{3}\pi^2 - \frac{3}{2}\ln\frac{x_1x_2}{\left(1-x_1\right)\left(1-x_2\right)} + 2\ln\frac{x_1}{1-x_1}\ln\frac{x_2}{1-x_2}\right]\\
	&+\frac{1}{2} A \frac{\delta\left(t_2-x_2\right)}{\left(x_1+x_2\right)}\left[\frac{t_1^2+x_1^2}{t_1^2\left(t_1-x_1\right)_{+}} \ln\frac{\left(x_1+x_2\right)\left(1-x_2\right)}{x_2\left(t_1+x_2\right)} + \frac{3}{2\left(t_1-x_1\right)_{+}} -\frac{2}{t_1} - \frac{3x_1}{t_1^2}\right]\\
	&+\left(1\leftrightarrow 2\right) + \frac{1}{2} A \left[\frac{G^A\left(t_1,t_2\right)}{\left[\left(t_1-x_1\right)\left(t_2-x_2\right)\right]_{+}} +H^A\left(t_1,t_2\right)\right],
\end{split}
\end{equation}
where the functions $G^A$ and $H^A$ are given by
\begin{align}
	G^A\left(t_1,t_2\right) &= \frac{\left(t_1+t_2\right)\left(\tau^2+\left(t_1t_2\right)^2\right)}{\left(t_1t_2\right)^2\left(t_1+x_2\right)\left(t_2+x_1\right)},\\
	H^A\left(t_1,t_2\right) &= \frac{-2}{t_1t_2\left(t_1+t_2\right)}.
\end{align}
And
\begin{equation}
	A=\frac{16\alpha^2\alpha_s}{27Q^2s}, \quad \tau=x_1x_2.
\end{equation}

Similarly, the cross section for the Compton scattering process (\cref{subfig:DY_gc}) is given by 
\begin{equation}
	\frac{d^2\sigma^C}{dQ^2dx_{F}} = \sum_q e^2_q \int^1_{x_1} \dd{t_1} \int^1_{x_2} \dd{t_2}
	\frac{d^2\hat{\sigma}^{C}}{dQ^2dx_F} f_{g/A}\left(t_1\right)
	\left[f_{q/B}\left(t_2\right) +f_{\bar{q}/B}\left(t_2\right) \right] + \left(1,A\leftrightarrow 2,B\right),
	\label{eq:compton_full}
\end{equation}
where the partonic cross section is given by
\begin{equation}
	\begin{split}
		\frac{d^2\hat{\sigma}^{C}}{dQ^2dx_F} =&\frac{3}{16} A \frac{\delta\left(t_2-x_2\right)}{\left(x_1+x_2\right)t_1^3}\left[ \left(x_1^2+\left(t_1-x_1\right)^2\right)\ln\frac{\left(x_1+x_2\right)\left(1-x_2\right)}{x_2\left(t_1+x_2\right)} - 6x_1\left(t_1-x_1\right)+t_1^2\right]\\
		&+\frac{3}{16}A\left[\frac{G^C\left(t_1,t_2\right)}{\left(t_2-x_2\right)_{+}} + H^C \left(t_1,t_2\right) \right],
	\end{split}
\end{equation}
and
\begin{align}
	G^C\left(t_1,t_2\right) &= \frac{\tau^2+\left(t_1t_2-\tau\right)^2}{t_1^3t_2^2\left(t_2+x_1\right)},\\
	H^C\left(t_1,t_2\right) &= \frac{1}{\left(t_1t_2\right)^2\left(t_1+t_2\right)^2}\left[t_1\left(t_2+x_1\right)\left(t_2-x_2\right)+2\tau\left(t_1+t_2\right)\right].
\end{align}
Note that in the second term of \cref{eq:compton_full}, the indices 1 and 2 must also be exchanged in
the expression of $\flatfrac{d^2\hat{\sigma}^{C}}{dQ^2dx_F}$.

