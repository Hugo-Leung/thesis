% For copyright and license information, see uiucthesis2021.dtx and derivatives.
\documentclass[final]{uiucthesis2021}
\usepackage[utf8]{inputenc}
\usepackage[english]{babel}
\usepackage{csquotes}
\usepackage{microtype}
\usepackage{amsmath,amsthm,amssymb}
\usepackage[bookmarksdepth=3,linktoc=all,colorlinks=true,urlcolor=blue,
	linkcolor=blue,citecolor=blue,pdfusetitle]{hyperref}
\usepackage{url}
\usepackage[capitalize]{cleveref}
\usepackage[style=phys,%
	biblabel=brackets,%
	chaptertitle=false,pageranges=false,%
	maxnames=4,sorting=none,eprint,backref=true]{biblatex}
\usepackage{physics}
\usepackage{subcaption}
\usepackage{tikz}
\usepackage{listings}
\usepackage[separate-uncertainty=true, range-units=single,range-phrase=\,--\, ]{siunitx}
\usepackage[version=4]{mhchem}
\usepackage[compat=1.1.0]{tikz-feynman}
\usepackage{tensor}
\usepackage{ifdraft}
\usepackage{placeins}
\usepackage{tablefootnote}
\usepackage{adjustbox}
\ifoptionfinal{
	\usepackage[disable]{pdfcomment}
}{
	\usepackage[author=chleung,draft]{pdfcomment}
	\usepackage{lineno}
	\linenumbers
}
\usepackage{multirow}
\usepackage{graphicx}
\graphicspath{{images/}}
\usepackage{verbatim}
\addbibresource{references.bib}
\addbibresource{docdb.bib}
\setcounter{tocdepth}{2}
\usepackage{xr}
\usepackage{subfiles}
\externaldocument[M-]{main}
\DeclareSIUnit\barn{b}
\renewcommand{\thefootnote}{\fnsymbol{footnote}}
\begin{document}


\title{Probing Parton Distributions in Proton using Drell-Yan and Charmonium Production
	in \texorpdfstring{$p+p$}{p+p} and \texorpdfstring{$p+d$}{p+d} Interactions with
	\texorpdfstring{\SI{120}{\GeV}}{120~GeV} Proton Beam at Fermilab}
\author{Ching Him Leung}
\department{Physics}

\phdthesis
\degreeyear{2024}
\committee{
	Professor Anne M Sickles , Chair\\
	Professor Jen-Chieh Peng, Director of Research\\
	Associate Professor Jorge Leite Noronha Jr \\
	Professor Jun Song }
\maketitle
\frontmatter

\begin{abstract}
	E906/SeaQuest is a fixed-target experiment at Fermilab with a \SI{120}{\GeV} proton beam.
	The primary goal of the experiment is to study the partonic structure of the nucleon.
	Muon pairs with mass between \num{2} to \SI{9}{\GeV} from the
	interaction of proton beam with various targets,
	including liquid hydrogen and deuterium, have been detected.
	These events contains muons originating from the Drell-Yan process as well as charmonium decays.
	The measurement of the Drell-Yan $\sigma_{pd}/2\sigma_{pp}$ cross section
	ratio is particularly sensitive to the light sea-quark asymmetry in the proton.
	The SeaQuest measurement provides information on the $\bar{d}/\bar{u}$ asymmetry
	for  Bjorken-$x$ ranging from \num{0.1} to \num{0.45}.
	The improved statistics	of the SeaQuest measurement allows for a check on the surprising behavior of
	$\bar{d}/\bar{u}$ at large $x$ observed in E866/NuSea.
	The Drell-Yan cross section ratio is extracted from the entire SeaQuest dataset
	using two independent methods.
	The measured $\sigma_{pd}/2\sigma_{pp}$ ratio is found to remain about unity,
	indicating the $\bar{d}>\bar{u}$ for $x$ up to \num{0.45}.
	The new result is compared with various parton distribution functions and theoretical calculations.
	In contrast to the Drell-Yan process, the charmonium production can be used to probe both the quark
	content as well as the gluon content.
	While the charmonium production mechanism is not as well understood as the Drell-Yan process,
	the large $J/\psi$ production cross section allows for a higher statistics measurement.
	The SeaQuest charmonium production measurement covers the forward rapidity region of $0.5 < x_F <0.9$.
	The measured cross sections are found to be in good agreement with
	theoretical calculations based on nonrelativistic QCD (NRQCD).
	The $\sigma_{\psi^\prime} / \sigma_{J/\psi}$ cross section ratios are found to increase as $x_F$ increases,
	indicating that the $q\bar{q}$ annihilation process has larger contributions to the
	$\psi'$ production than the $J/\psi$ production.
	The $\sigma_{pd}/2\sigma_{pp}$ cross section ratios are observed to be significantly different between
	the Drell-Yan process and charmonium production, reflecting the different production mechanisms.
	The $P_T$ distributions for $J/\psi$ and $\psi^\prime$ production are also presented and
	compared with data collected at higher energies.
\end{abstract}


\ifoptionfinal{
	\subfile{section/ack}
}{}

{
	\hypersetup{linkcolor=black}  % disable link coloring locally
	\tableofcontents
	% the Graduate College doesn't recommend including lot or lof
	%\listoftables
	\listoffigures
}


\mainmatter

\subfile{section/1_Intro}

\subfile{section/2_seaquest}

\subfile{section/3_analysis}

\subfile{section/4_result}

%\subfile{section/5_spinquest} % comment out SpinQuest for now

\subfile{section/6_conclusion}

% per Graduate College preference, place the \appendix and the appendices content before the
% bibliography (here) only if the appendices contain references.
{
	\backmatter
	\printbibliography[heading=bibintoc,title={References}]
}
%\mainmatter
{
	\appendix
	\subfile{section/a1_kinematic}
	\subfile{section/a2_extrapolation}
	\subfile{section/a3_jpsi}
}
\end{document}
