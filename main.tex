% For copyright and license information, see uiucthesis2021.dtx and derivatives.
\documentclass{uiucthesis2021}
\usepackage[utf8]{inputenc}
\usepackage[english]{babel}
\usepackage{csquotes}
\usepackage{microtype}
\usepackage{amsmath,amsthm,amssymb}
\usepackage[bookmarksdepth=3,linktoc=all,colorlinks=true,urlcolor=blue,
	linkcolor=blue,citecolor=blue]{hyperref}
\usepackage{url}
\usepackage[capitalize]{cleveref}
\usepackage[style=phys,%
	biblabel=brackets,%
	chaptertitle=false,pageranges=false,%
	maxnames=4,sorting=none]{biblatex}
\usepackage{physics}
\usepackage{subcaption}
\usepackage{tikz}
\usepackage{listings}
\usepackage{siunitx}
\usepackage[version=4]{mhchem}
\usepackage[compat=1.1.0]{tikz-feynman}
\usepackage{tensor}
\usepackage{ifdraft}

\ifoptionfinal{
	\usepackage[disable]{pdfcomment}
}{
	\usepackage[author=chleung,draft]{pdfcomment}
	\usepackage{lineno}
	\linenumbers
}
\usepackage{graphicx}
\graphicspath{{images/}}
\addbibresource{references.bib}
\setcounter{tocdepth}{2}
\usepackage{xr}
\usepackage{subfiles}
\externaldocument[M-]{main}
\begin{document}


\title{Probing parton distributions in proton with charmonium production with
	\SI{120}{\GeV} proton beam at Fermilab \pdfmargincomment{need updating}}
\author{Ching Him Leung}
\department{Physics}

\phdthesis
\degreeyear{2023}
\committee{
	Associate Professor Anne M Sickles , Chair\\
	Professor Jen-Chieh Peng, Director of Research\\
	Professor \\
	Professor }
\maketitle
\frontmatter

\begin{abstract}
	E906/SeaQuest is a fixed-target experiment at Fermilab with a \SI{120}{\GeV}
	proton beam. Muon pairs with mass between \num{2} to \SI{9}{\GeV} from the
	interaction of proton beam with various targets has been detected. The primary
	goal of the experiment is to study the partonic structure of the nucleon. In
	particular, the charmonium production data can be used to probe both the quark
	content as well as the gluon content. The preliminary result from the analysis
	of the SeaQuest charmonium production data will be presented.  E1039/SpinQuest
	is a follow up experiment of SeaQuest. By utilizing a transversely polarized
	target,we could extend this study to the transverse momentum distribution of
	the partons.
	\pdfmargincomment{copied from prelim, needed to be rewritten}
\end{abstract}


%\subfile{section/ack}

{
\hypersetup{linkcolor=black}  % disable link coloring locally
\tableofcontents
% the Graduate College doesn't recommend including lot or lof
%\listoftables
\listoffigures
}


\mainmatter

\subfile{section/1_Intro}

\subfile{section/2_seaquest}

\subfile{section/3_analysis}

\subfile{section/4_result}

\subfile{section/5_spinquest}

\subfile{section/6_conclusion}

% per Graduate College preference, place the \appendix and the appendices content before the
% bibliography (here) only if the appendices contain references.
\backmatter

\printbibliography[heading=bibintoc,title={References}]


\end{document}
